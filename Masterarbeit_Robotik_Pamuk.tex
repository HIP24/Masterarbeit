%
% FH Technikum Wien
% !TEX encoding = UTF-8 Unicode
%
% Erstellung von Master- und Bachelorarbeiten an der FH Technikum Wien mit Hilfe von LaTeX und der Klasse TWBOOK
%
% Um ein eigenes Dokument zu erstellen, müssen Sie folgendes ergänzen:
% 1) Mit \documentclass[..] einstellen: Master- oder Bachelorarbeit, Studiengang und Sprache
% 2) Mit \newcommand{\FHTWCitationType}.. Zitierstandard festlegen (wird in der Regel vom Studiengang vorgegeben - bitte erfragen)
% 3) Deckblatt, Kurzfassung, etc. ausfüllen
% 4) und die Arbeit schreiben (die verwendeten Literaturquellen in Literatur.bib eintragen)
%
% Getestet mit TeXstudio mit Zeichenkodierung ISO-8859-1 (=ansinew/latin1) und MikTex unter Windows
% Zu beachten ist, dass die Kodierung der Datei mit der Kodierung des paketes inputenc zusammen passt!
% Die Kodierung der Datei twbook.cls MUSS ANSI betragen!
% Bei der Verwendung von UTF8 muss dnicht nur die Kodierung des Dokuments auf UTF8 gestellt sein, sondern auch die des BibTex-Files!
%
% Bugreports und Feedback bitte per E-Mail an latex@technikum-wien.at
%
% Versionen
% *) V0.7: 9.1.2015, RO: Modeline angepasst und verschoben
% *) V0.6: 10.10.2014, RO: Weitere Anpassung an die UK
% *) V0.5: 8.8.2014, WK: Literaturquellen überarbeitet und angepasst
% *) V0.4: 4.8.2014, WK: Initalversion in SVN eingespielt
%
\documentclass[MMR,Master,english]{twbook}
\usepackage[utf8]{inputenc}
\usepackage[T1]{fontenc}

%
% Hier biblatex & Biber konfigurieren; Vergessen Sie nicht, dass Sie biber verwenden müssen um eine Bibliothek zu erzeugen
%
\usepackage[backend=biber, style=numeric]{biblatex}
\addbibresource{Literatur.bib}

%
% Bei Bedarf bitte hier die Syntax-Highlightings anpassen
%
\usepackage[final]{listings}
\lstset{captionpos=b, numberbychapter=false,caption=\lstname,frame=single, numbers=left, stepnumber=1, numbersep=2pt, xleftmargin=15pt, framexleftmargin=15pt, numberstyle=\tiny, tabsize=3, columns=fixed, basicstyle={\fontfamily{pcr}\selectfont\footnotesize}, keywordstyle=\bfseries, commentstyle={\color[gray]{0.33}\itshape}, stringstyle=\color[gray]{0.25}, breaklines, breakatwhitespace, breakautoindent}
\lstloadlanguages{[ANSI]C, C++, [gnu]make, gnuplot, Matlab}

%Formatieren des Quellcodeverzeichnisses
\makeatletter
% Setzen der Bezeichnungen für das Quellcodeverzeichnis/Abkürzungsverzeichnis in Abhängigkeit von der eingestellten Sprache
\providecommand\listacroname{}
\@ifclasswith{twbook}{english}
{%
    \renewcommand\lstlistingname{Code}
    \renewcommand\lstlistlistingname{List of Code}
    \renewcommand\listacroname{List of Abbreviations}
}{%
    \renewcommand\lstlistingname{Quellcode}
    \renewcommand\lstlistlistingname{Quellcodeverzeichnis}
    \renewcommand\listacroname{Abkürzungsverzeichnis}
}
% Wenn die Option listof=entryprefix gewählt wurde, Definition des Entyprefixes für das Quellcodeverzeichnis. Definition des Macros listoflolentryname analog zu listoflofentryname und listoflotentryname der KOMA-Klasse
\@ifclasswith{scrbook}{listof=entryprefix}
{%
    \newcommand\listoflolentryname\lstlistingname
}{%
}
\makeatother
\newcommand{\listofcode}{\phantomsection\lstlistoflistings}

% Die nachfolgenden Pakete stellen sonst nicht benötigte Features zur Verfügung
\usepackage{blindtext}

%
% Einträge für Deckblatt, Kurzfassung, etc.
%
\title{Virtualisierung eines Echtzeit-Betriebssystems zur Steuerung eines Roboters mit Schwerpunkt auf die
Einhaltung der Echtzeit}
\author{Halil Pamuk, BSc}
\studentnumber{51842568}
%\author{Titel Vorname Name, Titel\and{}Titel Vorname Name, Titel}
%\studentnumber{XXXXXXXXXXXXXXX\and{}XXXXXXXXXXXXXXX}
\supervisor{Sebastian Rauh, MSc. BEng}
%\supervisor[Begutachter]{Titel Vorname Name, Titel}
%\supervisor[Begutachterin]{Titel Vorname Name, Titel}
%\secondsupervisor{Titel Vorname Name, Titel}
%\secondsupervisor[Begutachter]{Titel Vorname Name, Titel}
%\secondsupervisor[Begutachterinnen]{Titel Vorname Name, Titel}
\place{Wien}
\kurzfassung{Erstellung einer Echtzeit-Robotersteuerungsplattform unter Verwendung von Salamander OS, Xenomai, QEMU
und PCV-521 in der Yocto-Umgebung. Die Plattform basiert auf Salamander OS und nutzt Xenomai für Echtzeit-
Funktionen. Dazu muss im ersten Schritt die Virtualisierungsplattform evaluiert werden. (QEMU, Hyper-V, Virtual
Box, etc.) Als weiterer Schritt folgt die Anbindung eines Roboters über eine VARAN-Bus Schnittstelle. Das
gesamte System wird in der Yocto-Umgebung erstellt und konfiguriert.
Das Hauptziel der Arbeit ist es, herauszufinden, wie die Integration von Echtzeit-Funktionen und effizienten
Kommunikationssystemen in eine Robotersteuerungsplattform die Reaktionszeit und Zuverlässigkeit von
Roboteranwendungen verbessern kann}
\schlagworte{Schlagwort1, Schlagwort2, Schlagwort3, Schlagwort4}
\outline{\blindtext}
\keywords{Echtzeit, Virtualisierung, Xenomai, VARAN}
%\acknowledgements{\blindtext}

\begin{document}

\maketitle

%
% .. und hier beginnt die eigentliche Arbeit. Viel Erfolg beim Verfassen!
%
%Die folgende Gliederung von Bachelor- und Masterarbeiten ist für die Robotik-Studiengänge
%verpflichtend vorgeschrieben.
%
%- Deckblatt
%- Eidesstattliche Erklärung (digital signiert)
%- Kurzfassung und Abstract mit Schlüsselwörtern/Keywords
%- Danksagung (optional)
%- Inhaltsverzeichnis

%- Einleitung (die genannten Punkte müssen keine eigenen Unterkapitel sein, müssen aber aus der Einleitung klar hervorkommen):
%   - Stand der Technik
%   - Problem- und Aufgabenstellung
%   - Zielsetzung
%- Hauptteil -> Gliederung je nach Thema
%   - Methodik
%   - Resultate
%   - Wirtschaftliche Betrachtung (optional)
%   - Diskussion
%- Zusammenfassung und Ausblick

%- Literaturverzeichnis
%- Abbildungsverzeichnis
%- Tabellenverzeichnis
%- Abkürzungsverzeichnis (optional)
%- Glossar (optional)
%- Anhang (optional)
\chapter{Einleitung}
\blindmathpaper
\clearpage
\section{Stand der Technik}
\blindtext
\clearpage
\section{Problem- und Aufgabenstellung}
\blindtext
\clearpage
\section{Zielsetzung}
\blindtext

%%%%%%%%%%%%%%%%%%%%%%%%%%%%%%%%%%%%%%%%%%%%%%%%%%%%%%%%%%
\clearpage
\chapter{Methodik}

\noindent Querverweise werden in \LaTeX{} automatisch erzeugt und verwaltet, damit sie leicht aktualisiert werden können. Hier wird zum Beispiel auf Abbildung \ref{Abb1} verwiesen.

\begin{figure}[!htbp]
\centering
\includegraphics[width=0.5\linewidth]{img/buchruecken}
\caption{Beispiel für die Beschriftung eines Buchrückens.}\label{Abb1}
\end{figure}
\begin{figure}[!htbp]
\centering
\includegraphics[width=0.5\linewidth]{img/buchruecken}
\caption{2. Beispiel für die Beschriftung eines Buchrückens.}\label{Abb2}
\end{figure}

Und hier ist ein Verweis auf Tabelle \ref{tab1}. Das gezeigte Tabellenformat ist nur ein Beispiel. Tabellen können individuell gestaltet werden.

\begin{table}[!htbp]
\centering
\caption{Semesterplan der Lehrveranstaltung \glqq Angewandte Mathematik\grqq.}\label{tab1}
\begin{tabular}{| p{0.3\linewidth} | p{0.3\linewidth} | p{0.3\linewidth} |}\hline
Datum & Thema & Raum\\\hline
20.08.2008 & Graphentheorie & HS 3.13\\
01.10.2008 & Biomathematik & HS 1.05\\\hline
\end{tabular}
\end{table}
\begin{table}[!htbp]
\centering
\caption{2. Semesterplan der Lehrveranstaltung \glqq Angewandte Mathematik\grqq.}\label{tab2}
\begin{tabular}{| p{0.3\linewidth} | p{0.3\linewidth} | p{0.3\linewidth} |}\hline
Datum & Thema & Raum\\\hline
20.08.2008 & Graphentheorie & HS 3.13\\
01.10.2008 & Biomathematik & HS 1.05\\\hline
\end{tabular}
\end{table}

Hier wird auf die Formel \ref{Gl1} verwiesen.

\begin{align}
x = -\frac{p}{2}\pm\sqrt{\frac{p^2}{4}-q}\label{Gl1}
\end{align}
\begin{align}
x = -\frac{p}{2}\pm\sqrt{\frac{p^2}{4}-q}\label{Gl2}
\end{align}

\begin{lstlisting}[language=C++,name={1. Beispiel},label={sc:bsp:1}]
#include <iostream>

void SayHello(void)
{
    // Kommentar
    cout << "Hello World!" << endl;
}

int main(int argc, char **argv)
{
    SayHello();
    return 0;
}
\end{lstlisting}

Literaturverweise sollten automatisch verwaltet werden, vor allem, wenn es viele Quellenverweise gibt. Beispiele sind  \cite{Ko05a}, \cite{Ko05b}, \cite{MiGo05}, \cite{TeGo14}, \cite{HuHa07}, \cite{HuZi10}, \cite{ZiKu07}, \cite{He07}, \cite{SIE11}, \cite{SIE14}, \cite{ISO98}, \cite{ATM11}, \cite{Hu11}, \cite{Po10}. Das verwendete Zitierformat (bzw.~das Format des Literaturverzeichnisses) ist entspechend der Vorgaben der Studiengänge zu wählen.
Es wird dringend empfohlen, BibTeX~zu verwenden (wie in diesem Beispiel).


\clearpage
\chapter{Hauptteil}
\clearpage
\chapter{Resultate}
\clearpage
\chapter{Diskussion}
\clearpage
\chapter{Zusammenfassung und Ausblick}



%
% Hier beginnen die Verzeichnisse.
%
\clearpage
\printbibliography
\clearpage

% Das Abbildungsverzeichnis
\listoffigures
\clearpage

% Das Tabellenverzeichnis
\listoftables
\clearpage

% Das Quellcodeverzeichnis
\listofcode
\clearpage

\phantomsection
\addcontentsline{toc}{chapter}{\listacroname}
\chapter*{\listacroname}
\begin{acronym}[XXXXX]
    \acro{ABC}[ABC]{Alphabet}
    \acro{WWW}[WWW]{world wide web}
    \acro{ROFL}[ROFL]{Rolling on floor laughing}
\end{acronym}

%
% Hier beginnt der Anhang.
%
\clearpage
\appendix
\chapter{Anhang A}
\clearpage
\chapter{Anhang B}
\end{document}
